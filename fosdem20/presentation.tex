\documentclass{beamer}
\hypersetup{colorlinks=true}
\graphicspath{{./images/}}
\usefonttheme{professionalfonts}
\usefonttheme{serif}
\usepackage{fontspec}
\setmainfont{Roboto Slab}
\usepackage{multicol}

\title{Nuspell: version 3 of the new spell checker}
\subtitle{FOSS spell checker implemented in C++17 with aid of Mozilla}
\author{Sander van Geloven}
\institute{FOSDEM, Brussels}
\date{February 1, 2020}
\keywords{spell checking, spell checker, spelling, Nuspell}

\begin{document}

\begin{frame}
\titlepage
\end{frame}

\section*{Outline}
\begin{frame}
\tableofcontents
\end{frame}



\section{Nuspell}

\begin{frame}{Nuspell}
\begin{columns}
\begin{column}{0.5\textwidth}
Nuspell is
\\\mbox{}
\begin{itemize}
\item spell checker
\\\mbox{}
\item free and open source software with LGPL
\\\mbox{}
\item library and command-line tool
\\\mbox{}
\item written in C++17
\end{itemize}
\end{column}
\begin{column}{0.5\textwidth}
\begin{figure}
\def
\centering\svgwidth{.5\columnwidth}
\input{images/logo-nuspell.pdf_tex}
\end{figure}
\end{column}
\end{columns}
\end{frame}

\begin{frame}{Nuspell – Team}
Our team currently consists of\\\mbox{}
\begin{itemize}
\item Dimitrij Mijoski
\begin{itemize}
\item lead software developer
\item \href{https://github.com/dimztimz}{github.com/dimztimz}
\end{itemize}
\mbox{}
\item Sander van Geloven
\begin{itemize}
\item information analyst
\item \href{http://hellebaard.nl}{hellebaard.nl}
\item \href{https://linkedin.com/in/svgeloven}{linkedin.com/in/svgeloven}
\item \href{https://github.com/PanderMusubi}{github.com/PanderMusubi}
\end{itemize}
\end{itemize}
\end{frame}

\begin{frame}{Nuspell – Spell Checking}
Spell checking is \alert{not trivial}
\begin{itemize}
\item much more than searching a long flat word list
\item dependent of language, character encoding and locale
\item involves case conversion, affixing, compounding, etc.
\item suggestions for spelling, typing and phonetic errors
\item long history of decades with \texttt{spell}, \texttt{ispell}, \texttt{aspell}, \texttt{myspell}, \texttt{hunspell} and now \texttt{nuspell}
\end{itemize}
\mbox{}\\
See also my talks at FOSDEM 2019 and FOSDEM 2016 at\\
\href{https://archive.fosdem.org/2019/schedule/event/nuspell}{archive.fosdem.org/2019/schedule/event/nuspell} and\\
\href{https://archive.fosdem.org/2016/schedule/event/integrating\_spell\_and\_grammar\_checking}{archive.fosdem.org/2016/\\schedule/event/integrating\_spell\_and\_grammar\_checking}

\end{frame}

\begin{frame}{Nuspell – Goals}
Nuspell's goals are
\begin{itemize}
\item a drop-in replacement for browsers, office suites, etc.
\item backwards compatibility MySpell and Hunspell format
\item improved maintainability
\item minimal dependencies
\item maximum portability
\item improved performance
\item suitable for further development and optimizations
\end{itemize}
\mbox{}\\
Implemented in object-oriented, templated and modern C++.
\end{frame}

\begin{frame}{Nuspell – Features}
Nuspell supports
\begin{itemize}
\item many character encodings
\item compounding
\item affixing
\item complex morphology
\item suggestions
\item personal dictionaries
\item 167 (regional) languages via 89 existing dictionaries
\end{itemize}
\end{frame}

\begin{frame}{Nuspell – Support}
Mozilla Open Source Support (MOSS) funded in 2018 the creation of Nuspell. Thanks to Gerv Markham$^†$ and Mehan Jayasuriya. See \href{https://mozilla.org/moss}{mozilla.org/moss} for more information.

Mozilla Open Source Support (MOSS) funded in 209/2020 the development of Nuspell version 3. Thanks to Mehan Jayasuriya and Bas Schouten. See \href{https://mozilla.org/moss}{mozilla.org/moss} for more information.


\begin{figure}
\centering
\def\svgwidth{.75\columnwidth}
\input{images/logo-mozilla.pdf_tex}
\end{figure}
Verification Hunspell has a mean precision of 1.000 and accuracy of 0.997. Perfect match 70\% of tested languages. On average checking 30\% faster and suggestions 8× faster.
\end{frame}



\section{Workings}

\begin{frame}{Workings – Spell Checking}
Spell checking is \alert{highly complex} and unfortunately not suitable for a lightning talk. It mainly concerns
\begin{itemize}
\item searching strings
\item using simple regular expressions
\item locale-dependent case detection and conversion
\item finding and using break patterns
\item performing input and output conversions
\item matching, stripping and adding (multiple) affixes, mostly in reverse
\item compounding in several ways, mostly in reverse
\item locale-dependent tokenization of plain text
\end{itemize}
\end{frame}

\begin{frame}{Workings – Case Conversion}
Examples of non-trivial case detection and conversion
\begin{itemize}
\item \texttt{to\_title("\alert{i}stanbul")} $\rightarrow$ \hfill English \texttt{"\alert{I}stanbul"}\\
\hfill Turkish \texttt{"\alert{İ}stanbul"}\\
\texttt{to\_upper("D\alert{i}yarbak\alert{ı}r")} $\rightarrow$ \hfill English \texttt{"D\alert{I}YARBAK\alert{I}R"}\\
\hfill Turkish \texttt{"D\alert{İ}YARBAK\alert{I}R"}\\
\item \texttt{to\_upper("{\fontspec{DejaVu Sans Mono}\alert{σ}ίγμα}")} $\rightarrow$ \hfill Greek \texttt{"{\fontspec{DejaVu Sans Mono}\alert{Σ}ΙΓΜΑ}"}\\
\texttt{to\_upper("{\fontspec{DejaVu Sans Mono}\alert{ς}ίγμα}")} $\rightarrow$ \hfill Greek \texttt{"{\fontspec{DejaVu Sans Mono}\alert{Σ}ΙΓΜΑ}"}\\
\texttt{to\_lower("{\fontspec{DejaVu Sans Mono}\alert{Σ}ΙΓΜΑ}")} $\rightarrow$ \hfill Greek \texttt{"{\fontspec{DejaVu Sans Mono}\alert{ς}ίγμα}"}\\
\item \texttt{to\_upper("Stra\alert{ß}e"} $\rightarrow$ \hfill English \texttt{Stra\alert{ß}e"}\\
\texttt{to\_upper("Stra\alert{ß}e"} $\rightarrow$ \hfill German \texttt{STRA\alert{SS}E"}
\item \texttt{to\_title("\alert{ij}sselmeeer")} $\rightarrow$ \hfill English \texttt{"\alert{Ij}sselmeer"}\\
\texttt{to\_title("\alert{ij}sselmeeer")} $\rightarrow$ \hfill Dutch \texttt{"\alert{IJ}sselmeer"}
\end{itemize}
\end{frame}

\begin{frame}{Workings – Suggestions}
\begin{enumerate}
\item upper case* \hfill \texttt{TODOh\alert{[ëê]}llo $\rightarrow$ TODOh\alert{e}llo}
\item replacement table \hfill \texttt{h\alert{[ëê]}llo $\rightarrow$ h\alert{e}llo}
\item mapping table \hfill \texttt{he\alert{łło\$} $\rightarrow$ he\alert{llo}}
\item adjacent swap* \hfill \texttt{TODOh\alert{h}ello $\rightarrow$ TODOhello}
\item distant swap* \hfill \texttt{TODOh\alert{h}ello $\rightarrow$ TODOhello}
\item keyboard layout \hfill \texttt{h\alert{r}llo $\rightarrow$ h\alert{e}llo}
\item extra character \hfill \texttt{h\alert{h}ello $\rightarrow$ hello}
\item forgotten character \hfill \texttt{hllo $\rightarrow$ h\alert{e}llo}
\item move character* \hfill \texttt{TODOhell\alert{ø} $\rightarrow$ TODOhell\alert{o}}
\item bad character \hfill \texttt{hell\alert{ø} $\rightarrow$ hell\alert{o}}
\item doubled two characters* \hfill \texttt{TODOhell\alert{ø} $\rightarrow$ TODOhell\alert{o}}
\item two words suggest* \hfill \texttt{TODOhell\alert{ø} $\rightarrow$ TODOhell\alert{o}}
\item phonetic mapping \hfill \texttt{\alert{\^{}el}lo $\rightarrow$ \alert{hel}lo}
\end{enumerate}
\end{frame}

\begin{frame}{WorkingsTODO – Dependencies}
\begin{itemize}
\item \tiny \texttt{\alert{gspell}: balsa, corebird, evince, evolution, geary, gedit, gnome-builder, gnome-recipes, gnome-software, gnote, gtranslator, latexila, osmo, polari}
\item \tiny \texttt{\alert{ispell}: bgoffice, dsdo, eo-spell, eoconv, eog, eog-plugins, espa-nol, espctag, espeak, espeak-ng, espeakedit, espeakup, esperanza, espresso, esptool, hkgerman, hkl, hl-todo-el, hlbr, hlbrw, ifrench, ifrench-gut, ifrit, ifscheme, ifstat, iftop, ifupdown, ifupdown-extra, ifupdown-multi, ifupdown2, ifuse, igal2, igdiscover, igmpproxy, ignition-cmake, ignition-cmake2, ignition-common, ignition-fuel-tools, ignition-math2, ignition-math4, ignition-msgs, ignition-transport, ignore-me, igor, igraph, igtf-policy-bundle, ii, ii-esu, iio-sensor-proxy, ippl, ippusbxd, iprange, iproute2, iprutils, ips, ipset, ipsvd, iptables, iptables-converter, iptables-netflow, iptables-optimizer, iptables-persistent, iptotal, iptraf-ng, iptstate, iptux, iputils, ipv6calc, ipv6pref, ipv6toolkit, ipvsadm, ipwatchd, ipwatchd-gnotify, ipxe, ipxe-qemu-256k-compat, irda-utils, irqbalance, irssi, isc-dhcp, isl, ispellcat, isrcsubmit, isso, istack-commons, istgt, isync, italc, itamae, iucode-tool, iw, jackd2, m17n-lib, magyarispell, norwegian, petname, pkg-perl-tools, pkg-php-tools, pkgconf, pkgdiff, pkgsel, pkgsync, pktanon, pktools, pktstat, softcatala-spell}
\end{itemize}
\end{frame}

\begin{frame}{WorkingsTODO – Dependencies}
\begin{itemize}
\item \tiny \texttt{\alert{python3-enchant}: mwic, ocrfeeder, pygtkspellcheck, python-autobahn, python-avro, python-aws-requests-auth, python-aws-xray-sdk, python-axolotl, python-axolotl-curve25519, python-azure, python-azure-devtools, python-azure-storage, python-b2sdk, python-babelgladeextractor, python-backcall, python-backports-abc, python-backports-shutil-get-terminal-size, python-backports.tempfile, python-backports.weakref, python-base58, python-txaio, python-txosc, python-typeguard, python-typing, python-typing-extensions, python-tz, python-tzlocal, sphinxcontrib-spelling, translate-toolkit, translation-finder, translatoid, transmageddon, transmission, transmission-el, transmission-remote-cli, transmission-remote-gtk, transmissionrpc, transrate-tools, transtermhp, trapperkeeper-clojure, trapperkeeper-metrics-clojure, trapperkeeper-scheduler-clojure, trapperkeeper-status-clojure, trapperkeeper-webserver-jetty9-clojure, trash-cli, traverso, travis, trayer, tre, tree, tree-puzzle, tree-style-tab, virtaal}
\end{itemize}
\end{frame}

\begin{frame}{WorkingsTODO – Dependencies}
\begin{itemize}
\item \tiny \texttt{\alert{hunspell}: aegisub, calibre, chromium, chromium-browser, codeblocks, codelite, dict-af, dict-nr, dict-ns, dict-ss, dict-st, dict-tn, dict-ts, dict-ve, dict-xh, dict-zu, dictd, dutch, dv4l, dvbackup, dvbcut, dvblast, dvbsnoop, dvbstream, dvbstreamer, dvbtune, dvcs-autosync, dvd+rw-tools, dvd-slideshow, dvdauthor, dvdbackup, dwz, e2fsprogs, ebtables, ec2-hibinit-agent, ed, edk2, efibootmgr, efivar, eject, elfutils, emacs, emacsen-common, enchant, enchant-2, entrypoints, eo-spell, eoconv, eog, eog-plugins, espa-nol, espctag, espeak, espeak-ng, espeakedit, espeakup, esperanza, espresso, esptool, featherpad, focuswriter, ghostwriter, goldendict, gwaei, hspell, hunspell-bo, hunspell-dz, ifrench, ifrench-gut, ifrit, ifscheme, ifstat, iftop, ifupdown, ifupdown-extra, ifupdown-multi, ifupdown2, ifuse, igaelic, igal2, igdiscover, igerman98, igmpproxy, ignition-cmake, ignition-cmake2, ignition-common, ignition-fuel-tools, ignition-math2, ignition-math4, ignition-msgs, ignition-transport, ignore-me, igor, igraph, igtf-policy-bundle, ii, ii-esu, iio-sensor-proxy, iirish, ispell-czech, ispell-fi, ispell-fo, ispell-gl, ispell-lt, ispell-tl, ispell-uk, ispell.pt, ispellcat, isrcsubmit, isso, istack-commons, istgt, isync, italc, itamae, iucode-tool, iw, jackd2, libgtk2-spell-perl, libreoffice, libtext-hunspell-perl, link-grammar, lokalize, mudlet, mythes, nixnote2, norwegian, onboard, pkg-perl-tools, pkg-php-tools, pkgconf, pkgdiff, pkgsel, pkgsync, pktanon, pktools, pktstat, plume-creator, psi-plus, pyhunspell, qtvirtualkeyboard-opensource-src, scribus, scribus-ng, sigil, sonnet, sphinxcontrib-spelling, tea, texstudio, texworks, thunderbird}
\end{itemize}
\end{frame}

\begin{frame}{WorkingsTODO – Dependencies}
\begin{itemize}
\item \tiny \texttt{\alert{libenchant}: abiword, ayttm, bibledit-gtk, bluefish, claws-mail, empathy, evolution, fcitx, fcitx5, geany-plugins, geary, gnome-builder, gnome-subtitles, gspell, gtkhtml4.0, gtkspell, gtkspell3, java-gnome, kadu, kde4libs, kvirc, lifeograph, lyx, ogmrip, php7.1, php7.2, php7.3, pluma, psi, purple-plugin-pack, pyenchant, qtspell, stardict, subtitleeditor, sylpheed, webkit, webkit2gtk, webkitgtk, xneur
python3-enchant: mwic, ocrfeeder, pygtkspellcheck, python-autobahn, python-avro, python-aws-requests-auth, python-aws-xray-sdk, python-axolotl, python-axolotl-curve25519, python-azure, python-azure-devtools, python-azure-storage, python-b2sdk, python-babelgladeextractor, python-backcall, python-backports-abc, python-backports-shutil-get-terminal-size, python-backports.tempfile, python-backports.weakref, python-base58, python-txaio, python-txosc, python-typeguard, python-typing, python-typing-extensions, python-tz, python-tzlocal, sphinxcontrib-spelling, translate-toolkit, translation-finder, translatoid, transmageddon, transmission, transmission-el, transmission-remote-cli, transmission-remote-gtk, transmissionrpc, transrate-tools, transtermhp, trapperkeeper-clojure, trapperkeeper-metrics-clojure, trapperkeeper-scheduler-clojure, trapperkeeper-status-clojure, trapperkeeper-webserver-jetty9-clojure, trash-cli, traverso, travis, trayer, tre, tree, tree-puzzle, tree-style-tab, virtaal}
\end{itemize}
\end{frame}



\begin{frame}{Workings – Initialization}
Initialize Nuspell in four steps in C++\\\mbox{}
\begin{itemize}
\item find, get and load dictionary\\
\texttt{auto find = Finder::\alert{search\_all\_dirs\_for\_dicts}();\\
auto path = find.\alert{get\_dictionary\_path}("en\_US");\\
auto dic = Dictionary::\alert{load\_from\_path}(path);}
\\\mbox{}
\item associate currently active locale\\
\texttt{boost::locale::generator gen;\\
auto loc = gen("");\\
dic.\alert{imbue}(loc);}
\end{itemize}
\mbox{}\\
These steps are more simple when using the API.
\end{frame}

\begin{frame}{Workings – Usage}
Use Nuspell by simply calling to\\\mbox{}
\begin{itemize}
\item check spelling\\
\texttt{auto spelling = false;\\
spelling = dic.\alert{spell}(word);}
\\\mbox{}
\item find suggestions\\
\texttt{auto suggestions = List\_Strings();\\
dic.\alert{suggest}(word, suggestions);}
\end{itemize}
\end{frame}



\section{Technologies}

\begin{frame}{Technologies – Headers}
Headers used in build process
\\\mbox{}
\begin{itemize}
\item C++17 library\\
e.g. GNU Standard C++ Library\\
\href{https://gcc.gnu.org/}{libstdc++} ≥ 7.0
\\\mbox{}
\item Boost.Locale\\
C++ facilities for localization\\
\href{https://www.boost.org/doc/libs/1_69_0/libs/locale/doc/html/index.html}{boost-locale} ≥ 1.62
\end{itemize}
\end{frame}

\begin{frame}{Technologies – Libraries}
Libraries used in run-time
\\\mbox{}
\begin{itemize}
\item C++17 library\\
e.g. GNU Standard C++ Library\\
\href{https://gcc.gnu.org/}{libstdc++} ≥ 7.0
\\\mbox{}
\item Boost.Locale\\
C++ facilities for localization\\
\href{https://www.boost.org/doc/libs/1_69_0/libs/locale/doc/html/index.html}{boost-locale} ≥ 1.62
\\\mbox{}
\item International Components for Unicode (ICU)\\
a C++ library for Unicode and locale support\\
\href{http://site.icu-project.org/}{icu} ≥ 57.1
\end{itemize}
\end{frame}

\begin{frame}{Technologies – Improvements}
TODO
\\\mbox{}
\begin{itemize}
\item C++14 → C++17
\item dropped
\item default support for UTF8, dropping usage of imbue, locale and codecvt and Bost header file
\end{itemize}
\end{frame}

\begin{frame}{Technologies – Compilers}
Currently supported compilers to build Nuspell
\begin{itemize}
\item GNU GCC compiler \href{https://gcc.gnu.org/}{g++} ≥ 7.0
\item LLVM Clang compiler \href{https://clang.llvm.org/}{clang} ≥ 5.0
\item Microsoft Visual C++ compiler \href{https://docs.microsoft.com/en-us/cpp/}{MSVC} ≥ 2017
\end{itemize}
\mbox{}\\
Upcoming supported compilers
\begin{itemize}
\item MinGW with MSYS \href{http://mingw.org/}{mingw}
\item GNU GCC compiler 6.0 (backport)
\end{itemize}
\end{frame}

\begin{frame}{Technologies – Tools}
Tools used for development
\begin{itemize}
\item build tools such as Autoconf, Automake, Make, Libtool and pkg-config
\item QtCreator for development and debugging, also possible with gdb and other command-line tools
\item unit testing with Catch2
\item continuous integration with Travis for GCC and Clang and coming soon AppVeyor for MinGW
\item profiling with Callgrind, KCachegrind, Perf and Hotspot
\item API documentation generation with Doxygen
\item code coverage reporting with LCOV and genhtml
\end{itemize}
\end{frame}



\section{Upcomming}

\begin{frame}{Upcoming – Next Version}
\begin{multicols}{2}
Next version will have improved
\begin{itemize}
\item \alert{performance}
\item compounding
\item suggestions
\item API
\item command-line tool
\item documentation
\item testing
\end{itemize}
Nuspell will then also be
\begin{itemize}
\item migrated to CMake
\item integrated with web browsers
\item offering ports and packages
\item offering language bindings
\end{itemize}
\end{multicols}
\end{frame}

\begin{frame}{Upcoming – Ports and Packages}
\begin{multicols}{2}
Supported
\begin{itemize}
\item Ubuntu ≥ 19.10 (Eoan Ermine)
\item Debian ≥ 10 (Buster)
\end{itemize}
\mbox{}\\
Tested
\begin{itemize}
\item FreeBSD ≥ 11
\end{itemize}
\mbox{}\\
Help wanted
\begin{itemize}
\item Android
\item Arch Linux
\item CentOS
\item Fedora
\item Gentoo
\item iOS
\item Linux Mint
\item macOS
\item NetBSD
\item OpenBSD
\item openSUSE
\item Slackware
\item Windows
\item ...
\end{itemize}
\end{multicols}
\end{frame}

\begin{frame}{Upcoming – Language Bindings}
\begin{multicols}{2}
Supported
\begin{itemize}
\item C++
\item C
\end{itemize}
\mbox{}\\
Help wanted
\begin{itemize}
\item C\#
\item Go
\item Java
\item JavaScript
\item Lua
\item Objective-C
\item Perl
\item PHP
\item Ruby
\item Rust
\item Python
\item Scala
\item ...
\end{itemize}
\end{multicols}
\end{frame}

\begin{frame}{Upcoming – Miscellaneous}
Other ways to help are
\begin{itemize}
\item fix bugs in dictionaries and word lists
\item improve dictionaries and word lists
\item contribute word lists with errors and corrections
\item integrate Nuspell with IDEs, text editors and editors for HTML, XML, JSON, YAML, \TeX{}, etc.
\item integrate Nuspell with Enchant e.g. for GtkSpell
\item sponsor our team
\item join our team
\end{itemize}
\end{frame}

\begin{frame}{Upcoming – Info and Contact}
\begin{columns}
\begin{column}{0.5\textwidth}
\href{https://nuspell.github.io}{nuspell.github.io}
\\\mbox{}\\
\href{https://twitter.com/nuspell1}{twitter.com/nuspell1}
\\\mbox{}\\
\href{https://facebook.com/nuspell}{facebook.com/nuspell}
\\\mbox{}\\
\href{https://fosstodon.org/@nuspell}{fosstodon.org/@nuspell}
\end{column}
\begin{column}{0.5\textwidth}
Big thank you to Dimitrij.
\\\mbox{}\\
\href{mailto:sander@hellebaard.nl?subject=Nuspell}{Contact} us to support the development, porting and maintenance of Nuspell.
\\\mbox{}\\
Thanks for your attention.
\end{column}
\end{columns}
\end{frame}



\end{document}
